% Options for packages loaded elsewhere
\PassOptionsToPackage{unicode}{hyperref}
\PassOptionsToPackage{hyphens}{url}
%
\documentclass[
]{book}
\usepackage{amsmath,amssymb}
\usepackage{iftex}
\ifPDFTeX
  \usepackage[T1]{fontenc}
  \usepackage[utf8]{inputenc}
  \usepackage{textcomp} % provide euro and other symbols
\else % if luatex or xetex
  \usepackage{unicode-math} % this also loads fontspec
  \defaultfontfeatures{Scale=MatchLowercase}
  \defaultfontfeatures[\rmfamily]{Ligatures=TeX,Scale=1}
\fi
\usepackage{lmodern}
\ifPDFTeX\else
  % xetex/luatex font selection
\fi
% Use upquote if available, for straight quotes in verbatim environments
\IfFileExists{upquote.sty}{\usepackage{upquote}}{}
\IfFileExists{microtype.sty}{% use microtype if available
  \usepackage[]{microtype}
  \UseMicrotypeSet[protrusion]{basicmath} % disable protrusion for tt fonts
}{}
\makeatletter
\@ifundefined{KOMAClassName}{% if non-KOMA class
  \IfFileExists{parskip.sty}{%
    \usepackage{parskip}
  }{% else
    \setlength{\parindent}{0pt}
    \setlength{\parskip}{6pt plus 2pt minus 1pt}}
}{% if KOMA class
  \KOMAoptions{parskip=half}}
\makeatother
\usepackage{xcolor}
\usepackage{longtable,booktabs,array}
\usepackage{calc} % for calculating minipage widths
% Correct order of tables after \paragraph or \subparagraph
\usepackage{etoolbox}
\makeatletter
\patchcmd\longtable{\par}{\if@noskipsec\mbox{}\fi\par}{}{}
\makeatother
% Allow footnotes in longtable head/foot
\IfFileExists{footnotehyper.sty}{\usepackage{footnotehyper}}{\usepackage{footnote}}
\makesavenoteenv{longtable}
\usepackage{graphicx}
\makeatletter
\def\maxwidth{\ifdim\Gin@nat@width>\linewidth\linewidth\else\Gin@nat@width\fi}
\def\maxheight{\ifdim\Gin@nat@height>\textheight\textheight\else\Gin@nat@height\fi}
\makeatother
% Scale images if necessary, so that they will not overflow the page
% margins by default, and it is still possible to overwrite the defaults
% using explicit options in \includegraphics[width, height, ...]{}
\setkeys{Gin}{width=\maxwidth,height=\maxheight,keepaspectratio}
% Set default figure placement to htbp
\makeatletter
\def\fps@figure{htbp}
\makeatother
\setlength{\emergencystretch}{3em} % prevent overfull lines
\providecommand{\tightlist}{%
  \setlength{\itemsep}{0pt}\setlength{\parskip}{0pt}}
\setcounter{secnumdepth}{5}
\usepackage{booktabs}
\ifLuaTeX
  \usepackage{selnolig}  % disable illegal ligatures
\fi
\usepackage[]{natbib}
\bibliographystyle{plainnat}
\IfFileExists{bookmark.sty}{\usepackage{bookmark}}{\usepackage{hyperref}}
\IfFileExists{xurl.sty}{\usepackage{xurl}}{} % add URL line breaks if available
\urlstyle{same}
\hypersetup{
  pdftitle={Health Policy Data Science Lab Manual},
  hidelinks,
  pdfcreator={LaTeX via pandoc}}

\title{Health Policy Data Science Lab Manual}
\author{}
\date{\vspace{-2.5em}2024-01-22}

\begin{document}
\maketitle

{
\setcounter{tocdepth}{1}
\tableofcontents
}
\hypertarget{stanford-lab-manual}{%
\chapter{Stanford Lab Manual}\label{stanford-lab-manual}}

The \href{http://healthpolicydatascience.org/}{Health Policy Data Science Lab} at Stanford is a group of interdisciplinary researchers who develop and apply quantitative methods to solve problems in health policy, leveraging techniques from statistics, computer science, economics, epidemiology, and decision science.

This manual will evolve over time to incorporate contributions from Lab members and collaborators. We drew inspiration from the lab manuals of our colleagues, including \href{https://jadebc.github.io/lab-manual/index.html}{Jade Benjamin-Chung} and \href{https://poldracklab.org/labguide/}{Russ Poldrack}.

Feel free to draw from this manual (and please cite it if you do)!

This work is licensed under a Creative Commons Attribution-NonCommercial 4.0 International License.

\hypertarget{lab-culture}{%
\chapter{Lab Culture}\label{lab-culture}}

by Sherri Rose

The culture of the Health Policy Data Science Lab is inquisitive, collaborative, kind, and inclusive so that scholars can do their best work. Our group has high standards for rigor, quality, and ethics and helps each other learn to reach these objectives. Every team member is a human first, and their career goals will be centered and supported. We are committed to an environment that is free from discrimination and harassment.

\hypertarget{mutual-respect-in-the-lab}{%
\section{Mutual Respect in the Lab}\label{mutual-respect-in-the-lab}}

We strive for a culture of mutual respect in all of our communications, meetings, and policies. Please demonstrate respect for all Lab members by, for example, practicing active listening, speaking only for yourself and not others, and not dominating conversations. Mutual respect for each other's time includes being prepared (including drafting agendas), planning ahead, and starting/ending meetings at the scheduled time.

\hypertarget{respect-for-the-people-represented-in-study-data}{%
\section{Respect for the People Represented in Study Data}\label{respect-for-the-people-represented-in-study-data}}

A key value in the Lab is deep respect for the people and communities whose information is represented in the data we study. We engage seriously in learning about the policies, institutional structures, societal biases, and lived experiences that underlie these data sources.

This respect also involves careful attention to data privacy and protection, discussed in Section \ref{data}.

\hypertarget{working-hours}{%
\section{Working Hours}\label{working-hours}}

Lab members are encouraged to work efficiently and effectively on a schedule that works well for them. \textbf{We do not support a culture of overwork!}

Dr.~Rose does not expect you to be working in the evenings or on weekends and asks that you respect Lab members' evening and weekend time as well. This includes not expecting Dr.~Rose to review your work product, submit letters, or otherwise be available for typical work tasks outside business hours.

We embrace time off, breaks from meetings, and vacations!

\hypertarget{trainee-support-access-and-accomodations}{%
\section{Trainee Support, Access, and Accomodations}\label{trainee-support-access-and-accomodations}}

Your physical and mental health are incredibly important. Please familiarize yourself with the \href{https://studentaffairs.stanford.edu/mental-health-resources-stanford}{mental health} and \href{https://glo.stanford.edu/glo-services/crisis-assistance}{crisis assistance} resources available for students at Stanford as well as mental \href{https://postdocbenefits.stanford.edu/my-benefits/medical-coverage/mental-health-resources}{health resources} for postdoctoral scholars.

Stanford is committed to providing equal educational opportunities for disabled students. Disabled students are a valued and essential part of the Stanford community. If you experience disability, please register with the \href{https://oae.stanford.edu/}{Office of Accessible Education (OAE)}. Professional staff at OAE will evaluate your needs, support appropriate and reasonable accommodations, and prepare an Academic Accommodation Letter for faculty. If you already have an Academic Accommodation Letter, Stanford invites you to share your letter with your advisor. Academic Accommodation Letters should be shared at the earliest possible opportunity so we may partner with you and OAE to identify any barriers to access and inclusion that might be encountered in your experience.

\hypertarget{general-policies}{%
\chapter{General Policies}\label{general-policies}}

by Sherri Rose

It is important that each trainee and Dr.~Rose have a shared understanding of expectations regarding research in the Lab and academic progress in the trainee's degree or postdoctoral program. The policies in this chapter aim to aid in this goal.

\hypertarget{joining-the-lab}{%
\section{Joining the Lab}\label{joining-the-lab}}

Dr.~Rose keeps a page updated \href{http://drsherrirose.org/new-students-and-postdocs}{on her website} regarding whether she is taking new students, hiring postdocs, or available for dissertation committees. At Stanford, we've had students from many different graduate programs join the lab, including health policy, biomedical data science, computer science, and chemical engineering programs.

\hypertarget{recurring-meetings}{%
\section{Recurring Meetings}\label{recurring-meetings}}

Lab trainees meet individually with Dr.~Rose regularly. This is typically weekly or every other week depending on the needs of the trainee and their projects. Meetings are either 25 minutes or 50 minutes to allow for breaks between meetings. Dr.~Rose expects trainees to manage the meeting such that it ends on time.

In general, Dr.~Rose expects progress to be made between each meeting. Progress can include struggling with the material. Dr.~Rose cares that you are engaged and taking initiative to advance the project. We'll discuss what you learned and where you are still confused. When we identify areas where you have gaps, Dr.~Rose does expect you to invest in learning the required material and to follow through to fill those gaps. All trainees must be making satisfactory academic progress in line with the expectations of their graduate degree program.

If you do not need to meet a particular week (e.g., making steady progress and don't have questions, busy preparing for an exam and did not work on research that week, etc), please email Dr.~Rose in advance so she can efficiently reallocate the meeting time. Your physical and mental health are important. If you need to cancel a meeting for health or personal reasons, email Dr.~Rose. As much notice as is possible is helpful, but it is always better to cancel a meeting short notice than to attend when you are sick!

If there are repeated meeting cancellations, we should discuss the underlying reasons.

Please also see the below Section \ref{meeting-agendas} for details on recurring meeting agendas.

\hypertarget{meeting-agendas}{%
\section{Meeting Agendas}\label{meeting-agendas}}

Trainees must prepare an agenda prior to each recurring meeting with Dr.~Rose. Create a google doc (invite Dr.~Rose as editor) that you'll add to in reverse chronological order for each meeting with the information below included. Update the google doc by 11AM \emph{one business day} before our meeting. Repeatedly not creating agendas will result in cancelled meetings.

\textbf{{[}Meeting Date{]}}

\begin{itemize}
\tightlist
\item
  What has been completed since previous meeting:
\item
  Topics to discuss at the meeting:
\item
  What will be completed by the next meeting:
\end{itemize}

\hypertarget{individual-development-plans}{%
\section{Individual Development Plans}\label{individual-development-plans}}

If Dr.~Rose is your primary advisor, students and postdoctoral scholars should complete the Stanford Individual Development Plan (IDP) when joining the Lab and then annually thereafter (\href{https://oge.stanford.edu/academics/idp/forms/}{student forms}, \href{https://postdocs.stanford.edu/sites/default/files/opa_idp-initial_0.pdf}{initial form for postdocs}, \href{https://postdocs.stanford.edu/sites/default/files/opa_idp-annual_0.pdf}{annual form for postdocs}). This applies regardless of home department. Trainees should plan to check in on progress made toward IDP goals once a quarter. Dr.~Rose expects that trainees will be responsible for scheduling the annual IDP meetings and adding the IDP check-ins once a quarter to the agenda for an existing recurring meeting.

\hypertarget{academic-progress-in-graduate-degree-programs}{%
\section{Academic Progress in Graduate Degree Programs}\label{academic-progress-in-graduate-degree-programs}}

If you are a graduate student and Dr.~Rose is your primary advisor, please create and regularly update a shared document that includes major degree requirements (e.g., coursework, teaching, qualifying exams, etc) and completion status (e.g., completed, planned completion date). Include links at the top to the graduate degree handbook from your home department as well as key contacts in your home department who should be kept apprised of your academic progress. This document will be discussed quarterly along with the IDP check-ins, and students should add it to the agenda for an existing recurring meeting.

\hypertarget{registering-for-units}{%
\section{Registering for Units}\label{registering-for-units}}

Graduate students should discuss their plans to register for research units with Dr.~Rose each term (often BIOMEDIN 299 or HRP 399). Units should be taken credit/no credit and not for a letter grade. Permission codes are currently required to register for research units.

\hypertarget{deadlines}{%
\section{Deadlines}\label{deadlines}}

We aim to set ambitious yet feasible target deadlines for work product in a collaborative process. It is often the case that research takes longer than we expect, and an internal agreed-upon deadline is no longer possible. If you anticipate missing a deadline, contact Dr.~Rose. It is an expectation in the Lab that all members are proactive about discussing revised deadlines rather than waiting until after the deadline has passed. If a trainee is repeatedly missing deadlines, we should discuss the underlying reasons.

\hypertarget{lab-meetings-events}{%
\section{Lab Meetings \& Events}\label{lab-meetings-events}}

The Lab holds Lab Meetings and various types of events throughout the year, including lunches, data jamborees, coffee chats, and journal clubs. If you have ideas for events, suggestions are always welcome.

Food at Lab events is funded by the Lab and free to Lab attendees. We expect that trainees who RSVP and submit food orders will show up to the event, barring illness or personal situation. If you need to change your RSVP, please contact Dr.~Rose to help us avoid food waste.

\hypertarget{communication}{%
\section{Communication}\label{communication}}

We have a Lab \textbf{slack}. Lab members can search for ``HPDS Lab'' in the \href{https://stanford.enterprise.slack.com/}{Workspaces at Stanford} and request to join.

Please keep in mind our Lab philosophy on working hours in Section \ref{working-hours}. Do not assume that because you have sent a slack message or email that you should get an instant reply.

\hypertarget{funding}{%
\chapter{Funding}\label{funding}}

by Sherri Rose

\hypertarget{for-graduate-students}{%
\section{For Graduate Students}\label{for-graduate-students}}

\hypertarget{for-postdoctoral-scholars}{%
\section{For Postdoctoral Scholars}\label{for-postdoctoral-scholars}}

\hypertarget{active-lab-funding}{%
\section{Active Lab Funding}\label{active-lab-funding}}

The Health Policy Data Science Lab at Stanford currently has three active grants where Dr.~Rose is the Principal Investigator:

\begin{itemize}
\tightlist
\item
  \href{https://reporter.nih.gov/search/ghBZmjOA8UyUXCt3VGZAZg/project-details/10485672}{NIH Director's Pioneer Award}
\item
  \href{https://reporter.nih.gov/search/ghBZmjOA8UyUXCt3VGZAZg/project-details/10676234}{NLM R01 Grant}
\item
  \href{https://www.arnoldventures.org/grants/the-board-of-trustees-of-the-leland-stanford-junior-university-10}{Laura and John Arnold Foundation Grant}
\end{itemize}

Dr.~Rose is also Co-Principal Investigator on the following two grants:

\begin{itemize}
\tightlist
\item
  \href{https://impact.stanford.edu/organization/health-care-fairness-impact-lab}{Stanford Impact Labs Grant}
\item
  \href{https://hai.stanford.edu/2022-hoffman-yee-grant-recipients\#eae}{Stanford HAI Hoffman-Yee Grant}
\end{itemize}

\hypertarget{trainee-responsibilities-on-funded-grants}{%
\subsection{Trainee Responsibilities on Funded Grants}\label{trainee-responsibilities-on-funded-grants}}

\hypertarget{reporting-requirements}{%
\subsection{Reporting Requirements}\label{reporting-requirements}}

\hypertarget{hourly-work}{%
\subsubsection*{Hourly Work}\label{hourly-work}}
\addcontentsline{toc}{subsubsection}{Hourly Work}

\hypertarget{participating-in-interim-and-annual-reports}{%
\subsubsection*{Participating in Interim and Annual Reports}\label{participating-in-interim-and-annual-reports}}
\addcontentsline{toc}{subsubsection}{Participating in Interim and Annual Reports}

\hypertarget{conflict-of-interest-reporting}{%
\subsubsection*{Conflict of Interest Reporting}\label{conflict-of-interest-reporting}}
\addcontentsline{toc}{subsubsection}{Conflict of Interest Reporting}

\hypertarget{fellowship-resources-reporting}{%
\subsubsection*{Fellowship \& Resources Reporting}\label{fellowship-resources-reporting}}
\addcontentsline{toc}{subsubsection}{Fellowship \& Resources Reporting}

\begin{itemize}
\tightlist
\item
  For Other Support pages
\end{itemize}

\hypertarget{data}{%
\chapter{Data}\label{data}}

by Sherri Rose and Marika Cusick

To add:

\begin{itemize}
\tightlist
\item
  Human subjects training
\item
  link back to respect for persons
\item
  stress ASK before doing
\end{itemize}

\hypertarget{irbs}{%
\section{IRBs}\label{irbs}}

\hypertarget{data-at-stanford}{%
\section{Data at Stanford}\label{data-at-stanford}}

\hypertarget{starr-data}{%
\subsection{STARR Data}\label{starr-data}}

Lab experts on the STARR data include Marika Cusick.

\hypertarget{center-for-population-health-sciences}{%
\subsection{Center for Population Health Sciences}\label{center-for-population-health-sciences}}

The \href{https://med.stanford.edu/phs.html}{Center for Population Health Sciences (PHS)}\ldots{}

\hypertarget{medicare-data}{%
\subsubsection*{Medicare Data}\label{medicare-data}}
\addcontentsline{toc}{subsubsection}{Medicare Data}

Lab experts on Medicare data include Marissa Reitsma.

\hypertarget{american-family-cohort-registry}{%
\subsubsection*{American Family Cohort Registry}\label{american-family-cohort-registry}}
\addcontentsline{toc}{subsubsection}{American Family Cohort Registry}

The American Family Cohort (AFC) Registry was created and is updated by the American Board of Family Medicine (ABFM).

Lab experts on the ABFM AFC Cohort Registry data include Agata Foryciarz, Gabriela Basel, and Marika Cusick.

\hypertarget{bringing-data-to-stanford}{%
\section{Bringing Data to Stanford}\label{bringing-data-to-stanford}}

The key steps in bringing a new data set to Stanford include completing submitting an IRB and a data risk assessment review.

\hypertarget{data-risk-assessment}{%
\subsection{Data Risk Assessment}\label{data-risk-assessment}}

The \href{https://uit.stanford.edu/security/dra}{data risk assessment (DRA) review} process at Stanford must be followed to bring external data to Stanford. A summary of this process is included below. However, the DRA review process is subject to change and should be confirmed and followed as described on the DRA website.

\begin{itemize}
\tightlist
\item
  Review the \href{https://uit.stanford.edu/guide/riskclassifications}{Stanford Risk Classifications} to determine the level of risk of your requested data.
\item
  If the requested data are high risk, then you will need to submit a \href{https://uit.stanford.edu/security/dra}{DRA}. If you are not sure if the data are high risk, there is also a pre-screening form that helps assess whether a DRA form is necessary.
\item
  In the DRA form, you will need the following information:

  \begin{itemize}
  \tightlist
  \item
    Project information

    \begin{itemize}
    \tightlist
    \item
      Project leader contact information
    \item
      IRB information (if applicable)
    \item
      Funding source
    \item
      Any other relevant parties involved in the project (e.g., Stanford Health Care)
    \item
      Any other individuals who will be involved with the data
    \end{itemize}
  \item
    Who are you getting the data from? (third party)

    \begin{itemize}
    \tightlist
    \item
      Contact information (e.g., name and email address)
    \item
      Data flow diagram
    \item
      Are the data going in or out of the U.S.?
    \end{itemize}
  \item
    Brief description of the project and reason for needing this data source
  \item
    Brief description of the data source

    \begin{itemize}
    \tightlist
    \item
      Elements (e.g., lab results, diagnoses or procedures)
    \item
      Number of records
    \item
      Data dictionary (if available)
    \item
      Data source (e.g., institutions and individuals involved in producing the data)
    \item
      Whether the data are identified or de-identified and how are the data de-identified? (e.g., Safe Harbor method)
    \end{itemize}
  \end{itemize}
\item
  Await the DRA review. You may get follow-up questions from the University Privacy Office, such as:

  \begin{itemize}
  \tightlist
  \item
    How do you plan to store the data?
  \item
    Will Stanford data be used or shared?
  \item
    Will data be shared back with the third party?
  \end{itemize}
\end{itemize}

Lab experts on the DRA process include Marika Cusick.

\hypertarget{datasets}{%
\subsection{Datasets}\label{datasets}}

We are in the process of bringing the \href{https://repository.niddk.nih.gov/studies/cric/}{Chronic Renal Insufficiency Cohort Study (CRIC)} from NIDDK to Stanford.

\hypertarget{omop}{%
\section{OMOP}\label{omop}}

\hypertarget{data-sharing}{%
\section{Data Sharing}\label{data-sharing}}

Many of our studies involve secondary analyses of existing health databases. It is typically not permitted for us to share such data due to privacy considerations. Thus, we often created simulated data that has some similar properties to the health databases to share along with our code and published results.

\hypertarget{simulated-data}{%
\section{Simulated Data}\label{simulated-data}}

Many of our projects involve simulating data to test our methodology under situations where we know the underlying truth and because we cannot share certain health data due to privacy considerations. Simulating data is an important skill to learn.

Examples of detailed simulation studies designed by Lab alums include work from \href{https://onlinelibrary.wiley.com/doi/10.1111/biom.13863}{Irina Degtiar} and \href{https://onlinelibrary.wiley.com/doi/10.1111/biom.13206}{Anna Zink}.

\emph{Note: Creating simulated data is different than designing a microsimulation study. It can be confusing!}

\hypertarget{computing-resources}{%
\chapter{Computing Resources}\label{computing-resources}}

by Sherri Rose

\hypertarget{nero-gcp}{%
\section{Nero GCP}\label{nero-gcp}}

\hypertarget{sherlock}{%
\section{Sherlock}\label{sherlock}}

\hypertarget{software}{%
\section{Software}\label{software}}

Much of the software we use in the Lab is available open source for free. Please let Dr.~Rose know if there is non-free software that would be helpful for your research and we can likely purchase this with research funds.

\hypertarget{statistical-computing}{%
\subsubsection*{Statistical Computing}\label{statistical-computing}}
\addcontentsline{toc}{subsubsection}{Statistical Computing}

\begin{itemize}
\tightlist
\item
  R: \href{https://cran.r-project.org/}{Download Free}
\item
  RStudio: \href{https://posit.co/products/open-source/rstudio/}{Download Free}
\item
  Python: \href{https://www.python.org/downloads/}{Download Free}
\end{itemize}

\hypertarget{illustration}{%
\subsubsection*{Illustration}\label{illustration}}
\addcontentsline{toc}{subsubsection}{Illustration}

\begin{itemize}
\tightlist
\item
  OmniGraffle: \href{https://store.omnigroup.com/omnigraffle}{Purchase Subscription or One-Time Download}
\item
  Adobe Illustrator: \href{https://library.stanford.edu/branner/using-branner-library}{Free via Stanford Branner Earth Science Library} or \href{https://web.stanford.edu/dept/its/cgi-bin/services/software/portal/detail.php?action=view_product\&product_id=1163}{Purchase Adobe Creative Cloud License}
\end{itemize}

\hypertarget{github}{%
\section{Github}\label{github}}

\hypertarget{conferences}{%
\chapter{Conferences}\label{conferences}}

by Sherri Rose

Conferences provide an opportunity to learn about new research, get feedback on your own research, practice presentation skills, take a short course, network, and more. Dr.~Rose is happy to discuss conferences relevant to your research and career goals in our meetings.

If we have discussed a conference and Dr.~Rose is funding your trip, please send Dr.~Rose a projected budget for the trip \emph{before} purchasing plane tickets, registrations, or reserving hotels.

\hypertarget{conferences-by-field}{%
\section{Conferences by Field}\label{conferences-by-field}}

We include below a noncomprehensive list of conferences that may be of interest to Lab members.

\hypertarget{algorithmic-bias-fairness}{%
\subsubsection*{Algorithmic Bias \& Fairness}\label{algorithmic-bias-fairness}}
\addcontentsline{toc}{subsubsection}{Algorithmic Bias \& Fairness}

\begin{itemize}
\tightlist
\item
  \href{https://facctconference.org}{ACM FAccT}
\item
  \href{https://www.aies-conference.com/}{AAI/ACM AI, Ethics, \& Society}
\item
  \href{https://eaamo.org/}{ACM EAAMO}
\item
  \href{https://satml.org/}{IEEE SaTML}
\end{itemize}

\hypertarget{machine-learning-for-health}{%
\subsubsection*{Machine Learning for Health}\label{machine-learning-for-health}}
\addcontentsline{toc}{subsubsection}{Machine Learning for Health}

\begin{itemize}
\tightlist
\item
  \href{https://www.mlforhc.org/}{Machine Learning for Health Care}
\item
  \href{https://www.chilconference.org/}{CHIL}
\item
  \href{https://ml4health.github.io/}{Machine Learning for Health}
\end{itemize}

\hypertarget{statistics}{%
\subsubsection*{Statistics}\label{statistics}}
\addcontentsline{toc}{subsubsection}{Statistics}

\begin{itemize}
\tightlist
\item
  \href{https://ww2.amstat.org/meetings/ichps/2020/}{International Conference on Health Policy Statistics}
\item
  \href{https://www.enar.org/meetings/future.cfm}{ENAR}
\item
  \href{https://www.amstat.org/asa/meetings/Joint-Statistical-Meetings.aspx}{Joint Statistical Meetings}
\end{itemize}

\hypertarget{health-economics}{%
\subsubsection*{Health Economics}\label{health-economics}}
\addcontentsline{toc}{subsubsection}{Health Economics}

\begin{itemize}
\tightlist
\item
  \href{https://www.ashecon.org/conferences/}{ASHEcon}
\item
  \href{https://healtheconomics.org/}{iHEA}
\item
  \href{https://www.economicsofai.com/nber-conference-toronto-2023}{Economics of AI}
\end{itemize}

\hypertarget{health-policy}{%
\subsubsection*{Health Policy}\label{health-policy}}
\addcontentsline{toc}{subsubsection}{Health Policy}

\begin{itemize}
\tightlist
\item
  \href{https://academyhealth.org/events/2023-06/2023-annual-research-meeting}{AcademyHealth}
\item
  \href{https://www.ispor.org/conferences-education/conferences}{ISPOR}
\end{itemize}

\hypertarget{decision-science}{%
\subsubsection*{Decision Science}\label{decision-science}}
\addcontentsline{toc}{subsubsection}{Decision Science}

\begin{itemize}
\tightlist
\item
  \href{https://smdm.org/meetings}{SMDM}
\end{itemize}

\hypertarget{stanford-travel-policies}{%
\section{Stanford Travel Policies}\label{stanford-travel-policies}}

Flights \textbf{MUST} be purchased through Stanford Travel. They are not reimbursable if they are purchased any other way.

\textbf{Health Policy and Biomedical Data Science Students}: If you are presenting at a conference, submit for up to \$1,000 of those expenses to be reimbursed through the \href{https://oge.stanford.edu/financial/travel-grant-program/}{Biosciences Travel Grant Program}. You are eligible for one conference per year and these funds should be used first before research grant resources.

\hypertarget{publications}{%
\chapter{Publications}\label{publications}}

by Sherri Rose

\hypertarget{authorship}{%
\section{Authorship}\label{authorship}}

We aim to discuss authorship early in the research process and have continuing conversations regarding team member roles. Plans can change and contributions may evolve over time. If you have authorship questions during the process of working on a project, we want you to feel empowered to ask these questions! \textbf{It is an expectation in the Lab that no new authors are invited to join an existing Lab project (i.e., where Dr.~Rose is the lead PI) without the prior agreement of, at a minimum, Dr.~Rose and the first author(s).} The Lab follows the \href{http://www.icmje.org/recommendations/browse/roles-and-responsibilities/defining-the-role-of-authors-and-contributors.html}{ICMJE recommendations} regarding who is included as an author.

\hypertarget{preprints}{%
\section{Preprints}\label{preprints}}

\hypertarget{on-least-publishable-units}{%
\section{On Least Publishable Units}\label{on-least-publishable-units}}

\hypertarget{types-of-papers}{%
\section{Types of Papers}\label{types-of-papers}}

The structure and content of a manuscript varies by discipline and audience.

Example Lab publications by journal type: \href{https://onlinelibrary.wiley.com/doi/10.1111/biom.13863}{statistics}, \href{https://jamanetwork.com/journals/jamapsychiatry/fullarticle/2765490?guestAccessKey=8e0d777c-ac53-47e9-9387-722b788058a2\&utm_campaign=author_alert-jamanetwork\&utm_content=author-author_engagement\&utm_medium=email\&utm_source=jps\&utm_term=1m}{medical}, \href{https://www.healthaffairs.org/doi/full/10.1377/hlthaff.2015.1026}{health policy}, \href{https://onlinelibrary.wiley.com/doi/10.1111/1475-6773.13046}{health services research}, \href{https://www.journals.uchicago.edu/doi/10.1086/716199}{health economics}, \href{https://proceedings.mlr.press/v180/chapfuwa22a.html}{computer science conference paper}.

\hypertarget{journals}{%
\section{Journals}\label{journals}}

The journal lists below are not exhaustive and largely focus on outlets where Lab members have previously published their work. Decisions about where to submit manuscripts for publication are made collaboratively to balance the needs and priorities of the team with as much deference as possible to what is best for the trainee author(s).

\hypertarget{health-economics-policy}{%
\subsubsection*{Health Economics \& Policy}\label{health-economics-policy}}
\addcontentsline{toc}{subsubsection}{Health Economics \& Policy}

\href{https://www.journals.elsevier.com/journal-of-health-economics}{Journal of Health Economics}, \href{https://www.journals.uchicago.edu/toc/ajhe/current}{American Journal of Health Economics}, \href{https://www.healthaffairs.org/}{Health Affairs}, \href{https://www.hsr.org/}{Health Services Research}, \href{https://journals.lww.com/lww-medicalcare/pages/default.aspx}{Medical Care}, \href{https://journals.sagepub.com/home/mdm}{Medical Decision Making}, \href{https://jamanetwork.com/journals/jama-health-forum}{JAMA Health Forum}

\hypertarget{epidemiology-public-health}{%
\subsubsection*{Epidemiology \& Public Health}\label{epidemiology-public-health}}
\addcontentsline{toc}{subsubsection}{Epidemiology \& Public Health}

\href{https://academic.oup.com/aje}{American Journal of Epidemiology}, \href{https://journals.lww.com/epidem/pages/default.aspx}{Epidemiology}, \href{https://academic.oup.com/ije}{International Journal of Epidemiology}, \href{https://ajph.aphapublications.org/}{American Journal of Public Health}

\hypertarget{statistics-1}{%
\subsubsection*{Statistics}\label{statistics-1}}
\addcontentsline{toc}{subsubsection}{Statistics}

\href{https://www.tandfonline.com/toc/uasa20/current}{Journal of the American Statistical Association}, \href{https://onlinelibrary.wiley.com/journal/15410420}{Biometrics}, \href{https://academic.oup.com/biostatistics}{Biostatistics} (\emph{COI: Sherri is Co-Editor-in-Chief}), \href{https://onlinelibrary.wiley.com/journal/10970258}{Statistics in Medicine}, \href{https://journals.sagepub.com/home/smm}{Statistical Methods in Medical Research}

\hypertarget{clinical}{%
\subsubsection*{Clinical}\label{clinical}}
\addcontentsline{toc}{subsubsection}{Clinical}

\href{https://www.nejm.org/}{NEJM}, \href{https://jamanetwork.com/}{JAMA}, \href{https://jamanetwork.com/journals/jamainternalmedicine}{JAMA Internal Medicine}, \href{https://jamanetwork.com/journals/jamapsychiatry}{JAMA Psychiatry}

\hypertarget{health-informatics-digital-health}{%
\subsubsection*{Health Informatics \& Digital Health}\label{health-informatics-digital-health}}
\addcontentsline{toc}{subsubsection}{Health Informatics \& Digital Health}

\href{https://academic.oup.com/jamia}{JAMIA}, \href{https://informatics.bmj.com/}{BMJ Health \& Care Informatics}, \href{https://www.thelancet.com/journals/landig/home}{Lancet Digital Health}

  \bibliography{book.bib,packages.bib}

\end{document}
